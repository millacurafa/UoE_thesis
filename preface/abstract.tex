\chapter{Abstract}
\markboth{\MakeUppercase{Abstract}}{\MakeUppercase{Abstract}}

While Synthetic Biology represents a promising approach to solve real-world problems, the use of genetically modified organisms (GMO) is still a cause of legal and environmental concerns. Cell-free systems (CFS) are an emerging technology where cell extracts are used instead of genetically modified cells, thus, not presenting a "living prospect" applicable to current legal regulations. Since there is a need for development of novel systems using the cell-free approach and considering that most attempts have been focused on mimicking normal cell behaviour, this work has as its principal aim to generate modular cell-free and cell-based systems capable of not only detecting substances present in the environment, such as heavy metals or antibiotics, but also analysing them via the usage of engineered behaviours, such as logic gates. In vivo logic gates, for instance, have proven difficult to combine into larger devices. Here we present a cell-based logic system, ParAlleL, which decomposes a large genetic circuit into a collection of small subcircuits working in parallel, each subcircuit responding to a different combination of inputs. A final global output is then generated by a combination of the responses. Using ParAlleL, for the first time a completely functional 3-bit full adder and full subtractor were generated using \textit{Escherichia coli} cells, as well as a calculator-like display that shows a numeric result, from 0 to 7, when the proper 3-bit binary input is introduced into the system. This parallel approach facilitates the design of cell-based logic gates by the decomposition of complex processes into their maincomponents, avoiding the need for complex genetic engineering. Cell-free systems, on the other hand, have emerged as a possible solution but much work is needed to optimize their functionality and simplify their usage for Synthetic Biology. Here we present a transcription-only genetic circuit (TXO), which is independent of translation or post-translational maturation. RNA aptamers are used as reaction output allowing the generation of fast, reliable and simple-to-design transcriptional units. TXO cell-free reactions and their possible applications are shown to be a promising new tool for fast and simple bench-to-market genetic circuit and biosensor applications.

Additionally, this thesis presents a versatile cell-free system based on the master survivalist bacteria \textit{Cupriavidus metallidurans} CH34, capable of not only sensing environmental variables, such as heavy metals, but also synthesizing proteins and producing bioplastics. This novel cell-free chassis follows the discovery of the unstable genome that this bacterium carries, which is also explained here, offering novel possibilities of development considering the cell free approach. 